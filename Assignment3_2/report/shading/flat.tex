\subsection{Flat shading}

Flat shading is a computationally cheap way of achieving a shading effect of an
object in a 3D image. The theory behind is that each polygon gets the same
intensity value added to every pixel. Provided enough rasterization, it can
give a decent result, but the main advantage of flat shading compared to more
advanced methods like Phong and Gouraud shading is the smaller faster computations. 

With flat shading, one surface normal is assumed for each polygon, usually the
normal at the center of the surface, or an average of the vertex normals. This
one normal is then used in the lighting model, and the resulting intensity
applied to the entire surface.

The result of flat shading is dependant on the position of the light
source with respect to the object, but the surface is assumed to be
diffuse reflector, and as such the entire polygon will get the same amount of
intensity added. The intensity itself is calculated with the phong
illumination model calculated for the center of each face. The flat
shading assumes some facts about the setup, namely that both viewer and light
source is far enough away from the object so that the illumination can be
spread evenly (or constant) througout the entire face.
