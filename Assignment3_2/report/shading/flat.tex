\subsection{Flat shading}

Flat shading is a computationally cheap way of achieving a shading effect of an
object in a 3D image. The theory behind is that each polygon gets the same
intensity value added to every pixel. Provided enough rasterization, it can
give a decent result, but the main advantage of flat shading compared to more
advanced methods like Phong and Gouraud shading is the smaller faster computations. 

With flat shading, one surface normal is assumed for each polygon, usually the
normal at the center of the surface, or an average of the vertex normals. This
one normal is then used in the lighting model, and the resulting intensity
applied to the entire surface.
