\subsection{The ShadeFace method}

The assignment provides us with a lot of boilerplate code used in shading of
the cube faces. This section will attempt to clarify what is being done in the
provided code

The code is segregated in multiple sections separated by a comment line of dots.
The first section is declaration of variables. One important variable is the
shadeRes, which gives the resolution and therefore the dimensionality of the
shade. The next section projects the homogenious points from real world
coordinates to 2D coordinates. Next section defines a square that covers the
desired face (meaning from 0 to shadeRes in all dimension. Next section is
defines a homography from the texture to the projected points, meaning that
we can go from real world to 2D. In the next section three matrices for red,
green and blue channels respectively is set by our CalculateShadeMatrix
function. Next section applies the intensity values as a texture on top of the
image using the cv2 method warpperspective and the homography estimated
earlier. Next section simply converts the image from BGR 2 RGB and
extracts each colour channel with the cv2 split method. The next section first
creates a copy of one of the channel images. Next it set's them all to 0 to
create a mask that will be affected only by the points within the face. This
is done by taking our previously projected points and fill the polygon they
make up with the value 255. This is done with the cv2 method
fillconvexpoly, for speed most likely. The next uses this mask of the face to
first index each channel and perform a map on these values. The mapping
function multiplies the value of the original colour channels with the
overlay channels and applies normalization in the range of 0 to 255. The next
and final section stitches the channels back together to form a complete image
and transforms it back to BGR.
