\section{Augmented Cube} In this assignment we compare two methods of
calculating camera views.Both utilize that the camera is already calibrated,
(See earlier section for info about calibration) but uses different methods for
calculating the extrinsic parameters.  

\subsection{Method one, using homography}
This method is the same as in the last assignment, so we'll explain it very
briefly.  As K is known in our projection matrix K*[R|T] we need to calculate R
and T.  As we have an existing view and four points within that view, that
corresponds to four ponts in our current frame. we can calculate a homography
between the two planes and use that to calculate our existing projection matrix.
We do that simply by multiplying our existing matrix with the homography. cam2 =
H * cam1 in the last assignment we also used a homography to transfer the
extrinsic parameters to a new space.There we estimated the "Z" axis by finding
the cross product between the "X" and "Y" vectors, finding the orthogonal
vector. This time, the z axis is already calculated in the first camera, and are
"transfered" as well with the homography. Our results so far shows that the
projection of the z axis is not nearly as precise as when we use the second
method.

\subsection{Method two, direct calculation} The second method also utilizes that
the kamera matrix is already known. Where method one tracked the chessboard
patterne in two different views, this method only tracks points in the current
frame. The other chesboard points are calculated from knowledge about the
pattern. When we know the starting coordinate, the size of each square and
number of squares, the points can be calculated precisely from the decided
origin. cv2.solvePnP is used to estimate a camera position relative to the
surface.This relation effectively gives us the extrinsic vectors.We convert
those to 3x3 matrices (Rodrigues), and stack them together. Lastly we take the
dot product between the resulting matrix and our kamera matrix, ending up with
our new projection matrix.
